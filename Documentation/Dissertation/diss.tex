\documentclass[12pt,twoside,notitlepage]{report}

%packages
\usepackage{a4}
\usepackage{hyperref}
\usepackage{verbatim}
\usepackage{booktabs, longtable}
\usepackage{tabularx}
\usepackage{xcolor,colortbl}
\usepackage{hhline}
\usepackage{graphicx}
\usepackage{subcaption}
\usepackage{epstopdf}
\usepackage{listings}       % Source code listings
\usepackage[toc]{glossaries}
\usepackage{changebar}
\usepackage{amsmath}
\usepackage{rotating}
\usepackage{tikz}
\usepackage{pgfplots}
\usepackage{pgfplotstable}
\usepackage[utf8]{inputenc}
\usepackage{graphicx}

% listings style
\lstset{
  basicstyle=\footnotesize\tt,                                          % the size of the fonts that are used for the code
  breakatwhitespace=false,                                              % sets if automatic breaks should only happen at whitespace
  breaklines=true,                                                      % sets automatic line breaking
  prebreak = \raisebox{0ex}[0ex][0ex]{\ensuremath{\hookleftarrow}},     % add arrows to indicate line breaks added
  framesep = 5px,                                                       % add padding to the top/bottom of listings
  captionpos=b,                                                         % sets the caption-position to bottom
  extendedchars=true,                                                   % lets you use non-ASCII characters; for 8-bits encodings only, does not work with UTF-8
  frame=TB,                                                             % adds a frame around the code (frame at bot and top)
  language=Caml,                                                        % the language of the code
  showspaces=false,                                                     % show spaces everywhere adding particular underscores; it overrides 'showstringspaces'
  showstringspaces=false,                                               % underline spaces within strings only
  showtabs=false,                                                       % show tabs within strings adding particular underscores
  tabsize=4,                                                            % sets default tabsize to 4 spaces
  numbers=left,                                                         % puts numbers on the left side
  keywordstyle=\color[rgb]{0,0,1}\ttfamily,
  stringstyle=\color[rgb]{0.627,0.126,0.941}\ttfamily,
  commentstyle=\color[rgb]{0.133,0.545,0.133}\ttfamily,
  morecomment=[l][\color{magenta}]{\#}
}

% Graphics extensions used
\DeclareGraphicsExtensions{.pdf, .png, .jpg, .eps}

% ??
\newcolumntype{R}{>{\raggedleft\arraybackslash}X}

% Pick the bibliocraphy style
\bibliographystyle{plain}

% Styling for pgf plots
\pgfplotsset{compat=newest}
\pgfplotsset{grid style={gray!50}}
\pgfplotsset{minor grid style={dashed,gray!50}}

%colours for bar graphs
\pgfplotsset{
   /pgfplots/bar  cycle  list/.style={/pgfplots/cycle  list={%
        {blue,fill=blue!50!white,mark=none},%
        {orange,fill=orange!50!white,mark=none},%
        {green,fill=green!50!white,mark=none},%
        {yellow,fill=yellow!40!white,mark=none},%
        {red,fill=red!50!white,mark=none},%
     }
   },
}

\usepgfplotslibrary{units}

% to allow postscript inclusions
\input{epsf}                 
% On thor and CUS read top of file:
%     /opt/TeX/lib/texmf/tex/dvips/epsf.sty
% On CL machines read:
%     /usr/lib/tex/macros/dvips/epsf.tex

% glossaries
\makeglossaries

\raggedbottom                           % try to avoid widows and orphans
\sloppy
\clubpenalty1000%
\widowpenalty1000%

\addtolength{\oddsidemargin}{6mm}       % adjust margins
\addtolength{\evensidemargin}{-8mm}

\renewcommand{\baselinestretch}{1.1}    % adjust line spacing to make
                                        % more readable

\begin{document}

% glossary definitions
\newacronym{btb}{BTB} {Branch Target Buffer}
\newacronym{bp0}{BTB8} {branch predictor with BTB and tablesize 8}
\newacronym{bp1}{BTB512} {branch predictor with BTB and tablesize 512}
\newacronym{sat}{SAT} {branch predictor with 3-bit saturating counters}
\newacronym{lhbp}{LHBP} {local history branch predictor}
\newacronym{pc}{PC} {Program Counter}
\newacronym{alu}{ALU} {Arithmetic Logic Unit}
\newacronym{risc}{RISC} {Reduced Instruction Set Computer}
\newacronym{ISA}{ISA} {Instruction Set Architecture}
\newacronym{ipc}{IPC} {Instructions Per Cycle}
\newacronym{cpi}{CPI} {Cycles per Instruction}
\newacronym{fpga}{FPGA} {Field Programmable Logic Array}
\newacronym{alut}{ALUT}{Adaptive Look-Up Table}
\newacronym{alm}{ALM} {Adaptive Logic Module}

%%%%%%%%%%%%%%%%%%%%%%%%%%%%%%%%%%%%%%%%%%%%%%%%%%%%%%%%%%%%%%%%%%%%%%%%%%%%%%%%%%%%%%%%%%%%%%%%%%%%%%%%%%%%%%%%%%%%%%%%
% Title

% Name
\begin{flushright}
    \Large
    \hfill{\LARGE \bf Michael Painter}
\end{flushright}

% Title + imgs
\begin{center}
    \vfill

    \Huge{\bf Spectral Image Analysis for Medical Imaging}
    
    \bigskip
    \bigskip
    
    {Part II Computer Science Tripos} \\
    
    \bigskip
    
    {Churchill College, 2016\\}

    \bigskip 

    {\today}

    \bigskip

    \includegraphics[scale=0.099]{titleimg/camcrest}
    \includegraphics[scale=0.2792]{titleimg/chucrest}

    \vfill
\end{center}

\cleardoublepage





%%%%%%%%%%%%%%%%%%%%%%%%%%%%%%%%%%%%%%%%%%%%%%%%%%%%%%%%%%%%%%%%%%%%%%%%%%%%%%%%%%%%%%%%%%%%%%%%%%%%%%%%%%%%%%%%%%%%%%%%
% Proforma, table of contents and list of figures

\setcounter{page}{1}
\pagenumbering{roman}
\pagestyle{plain}

\chapter*{Proforma}

{\large
\begin{tabularx}{\textwidth}{l X}
Name:               & \bf Michael Painter                      \\
College:            & \bf Churchill College                     \\
Project Title:      & \bf Spectral Image Analysis for Medical Imaging \\
Examination:        & \bf Computer Science Part II Project Dissertation, May 2016        \\
Word Count:         & \bf 579\footnotemark[1]\\
Project Originator: & Dr Pietro Lio'              \\
Supervisor:         & Dr Pietro Lio' \& Dr Gianluca Ascolani       \\ 
\end{tabularx}
}

\footnotetext[1]{
    This word count was computed by {\tt texcount -total diss.tex }
}
\stepcounter{footnote}


\section*{Original aims of the project}
    *TODO*

\section*{Work completed}
    *TODO*

\section*{Special difficulties}
    None.
 
\newpage
\section*{Declaration}

I, Michael Painter of Churchill College, being a candidate for Part II of the Computer Science Tripos, hereby declare 
that this dissertation and the work described in it are my own work, unaided except as may be specified below, and that 
the dissertation does not contain material that has already been used to any substantial extent for a comparable 
purpose.

\bigskip
\bigskip
\bigskip
\bigskip

\leftline{\rule{6cm}{0.5pt}}
\leftline{Signed}

\bigskip
\bigskip
\bigskip
\bigskip

\leftline{\rule{4cm}{0.5pt}}
\leftline{Date}










%%%%%%%%%%%%%%%%%%%%%%%%%%%%%%%%%%%%%%%%%%%%%%%%%%%%%%%%%%%%%%%%%%%%%%%%%%%%%%%%%%%%%%%%%%%%%%%%%%%%%%%%%%%%%%%%%%%%%%%%
% the contents

\cleardoublepage
\setcounter{tocdepth}{1}
\tableofcontents










%%%%%%%%%%%%%%%%%%%%%%%%%%%%%%%%%%%%%%%%%%%%%%%%%%%%%%%%%%%%%%%%%%%%%%%%%%%%%%%%%%%%%%%%%%%%%%%%%%%%%%%%%%%%%%%%%%%%%%%%
% list of figures, tables and listings

\listoffigures

\listoftables

\lstlistoflistings










%%%%%%%%%%%%%%%%%%%%%%%%%%%%%%%%%%%%%%%%%%%%%%%%%%%%%%%%%%%%%%%%%%%%%%%%%%%%%%%%%%%%%%%%%%%%%%%%%%%%%%%%%%%%%%%%%%%%%%%%
% acknowledgements

\newpage
\section*{Acknowledgements}

I thank my supervisors, director of studies and tutor for their extensive support, especially for putting up with me 
throughout the year. [TODO - anymore thanks] %I would also like to thank Malavika Nair, Ioana Bica and Yanie de Nadaillac
% for their support, multiple proof readings without complaint and helping my through the good days and bad days. 
% Without any single one of you this project would not have happened.










%%%%%%%%%%%%%%%%%%%%%%%%%%%%%%%%%%%%%%%%%%%%%%%%%%%%%%%%%%%%%%%%%%%%%%%
% now for the chapters

\cleardoublepage        % just to make sure before the page numbering
                        % is changed

\setcounter{page}{1}
\pagenumbering{arabic}
\pagestyle{headings}















%%%%%%%%%%%%%%%%%%%%%%%%%%%%%%%%%%%%%%%%%%%%%%%%%%%%%%%%%%%%%%%%%%%%%%%%%%%%%%%%%%%%%%%%%%%%%%%%%%%%%%%%%%%%%%%%%%%%%%%%
% the introduction

\cleardoublepage
\chapter{Introduction}
    In this chapter we introduce the problem: what data we are given and what would we like to do with it? We will 
    suggest some possible methods that could be used to implement a solution. We describe how we might want to 
    evaluate the performance of the system. Finally we will outline the overall pipeline of the system.

    \section{What are we trying to do?}
        *TODO* \\

        We can briefly describe the system as taking an image with a spectrum of intensity values at each pixel and we wish to 
        produce a pixel labelling. A pixel labelling is... and is often referred to as an image segmentation.

    \section{Possible solutions}
        *TODO*

    \section{My approach}
        *TODO*

    \section{Evaluation of system performance}
        *TODO*










%%%%%%%%%%%%%%%%%%%%%%%%%%%%%%%%%%%%%%%%%%%%%%%%%%%%%%%%%%%%%%%%%%%%%%%%%%%%%%%%%%%%%%%%%%%%%%%%%%%%%%%%%%%%%%%%%%%%%%%%
% the preparation

\cleardoublepage
\chapter{Preparation}
    In this section we investigate multiple machine learning algorithms, and decide on which may be reasonable to use
    in this project. We then investigate noise models for images, how they can be applied to our `hyperspectral' case 
    and what can be implemented to reduce or eliminate the noise. 

    \section{A summary of machine learning algorithms}
        *TODO*
        \subsection{Method 1}
            *TODO*
            
        \subsection{Method 2}
            *TODO*

        \subsection{Method 3}
            *TODO*



    \section{Image noise and noise simulation}
    *TODO*



    \section{De-noising}
    *TODO*







%%%%%%%%%%%%%%%%%%%%%%%%%%%%%%%%%%%%%%%%%%%%%%%%%%%%%%%%%%%%%%%%%%%%%%%%%%%%%%%%%%%%%%%%%%%%%%%%%%%%%%%%%%%%%%%%%%%%%%%%
% the implementation

\cleardoublepage
\chapter{Implementation}
    In this chapter we elaborate on the implementations of the randomised forests and neural network solutions. 
    Similarly we discuss the de-noising aspect of the system. We then look at how each of the random forests and neural 
    networks were trained for the example data sets provided. We finally return to look at the overview of the system 
    and how all of the components fit together into a pipeline.

    \section{Randomised Forests}
    *TODO*

      \subsection{Training}
      *TODO*

      \subsection{Classification}
      *TODO*

    \section{Neural Networks}
    *TODO*

      \subsection{Training}
      *TODO*

      \subsection{Classification}
      *TODO*

    \section{De-noising}
    *TODO*

    %\section{Using the GPU using OpenCL}









%%%%%%%%%%%%%%%%%%%%%%%%%%%%%%%%%%%%%%%%%%%%%%%%%%%%%%%%%%%%%%%%%%%%%%%%%%%%%%%%%%%%%%%%%%%%%%%%%%%%%%%%%%%%%%%%%%%%%%%%
% the evaluation

\cleardoublepage
\chapter{Evaluation}
    In this section we will evaluate the performance of the implemented machine learning solutions, comparing them based 
    on the following metrics: accuracy of pixel labellings from a test set with known classifications prior, training 
    time (as a function of the size of the training sequence) and classification time (as a function of the size of the 
    image). I will also look at how each of the components of the system effects each of the metrics mentioned in 
    isolation.

    \section{Accuracy}
    *TODO*

    \section{Training time}
    *TODO*

    \section{Classification time}
    *TODO*

    \section{The effect of the de-noising component}
    *TODO*

    \section{The effect of the number of trees}
    *TODO*

    \section{The effect of the depth of trees}
    *TODO*

    \section{The effect of the randomness of trees}
    *TODO*

    % \section{Performance of de-blurring in OpenCL}
    % *TODO* - compare FFT and non-FFT methods of convolution etc

    % \section{Performance of classification time in OpenCL}
    % *TODO* - compare to classification time with OpenCL in encog turned on, also compare to non openCL time.
    % Compare the classification times with openCL on/off in encog.










%%%%%%%%%%%%%%%%%%%%%%%%%%%%%%%%%%%%%%%%%%%%%%%%%%%%%%%%%%%%%%%%%%%%%%%%%%%%%%%%%%%%%%%%%%%%%%%%%%%%%%%%%%%%%%%%%%%%%%%%
% the conclusion


\cleardoublepage
\chapter{Conclusion}
    *TODO* - Summarise how the decision forests and neural networks compared (accuracy on the test data etc), basically 
    discuss the results from the evaluation.

    \section{Section 1}
    *TODO*

    \section{Section 2}
    *TODO*

    \section{Section 3}
    *TODO* \cite{Luthman:2015:hyperspectralImager}










%%%%%%%%%%%%%%%%%%%%%%%%%%%%%%%%%%%%%%%%%%%%%%%%%%%%%%%%%%%%%%%%%%%%%%%%%%%%%%%%%%%%%%%%%%%%%%%%%%%%%%%%%%%%%%%%%%%%%%%%
% the bibliography

\cleardoublepage
\addcontentsline{toc}{chapter}{Bibliography}
\bibliography{refs}
%\printbibliography
\cleardoublepage










%%%%%%%%%%%%%%%%%%%%%%%%%%%%%%%%%%%%%%%%%%%%%%%%%%%%%%%%%%%%%%%%%%%%%%%%%%%%%%%%%%%%%%%%%%%%%%%%%%%%%%%%%%%%%%%%%%%%%%%%
% the appendices
\appendix

%%%%%%%%%%%%%%%%%%%%%%%%%%%%%%%%%%%%%%%%%%%%%%%%%%%%%%%%%%%%%%%%%%%%%%%%%%%%%%%%%%%%%%%%%%%%%%%%%%%%%%%%%%%%%%%%%%%%%%%%
% description of benchmark tests

\cleardoublepage
\chapter {Benchmark descriptions}
    *TODO*

    \section{Section 1}
    *TODO*

    \section{Section 2}
    *TODO*

    \section{Section 3}
    *TODO*










%%%%%%%%%%%%%%%%%%%%%%%%%%%%%%%%%%%%%%%%%%%%%%%%%%%%%%%%%%%%%%%%%%%%%%%%%%%%%%%%%%%%%%%%%%%%%%%%%%%%%%%%%%%%%%%%%%%%%%%%
% evaluation data

\cleardoublepage
\chapter{Evaluation results}
    *TODO*

    \section{Section 1}
    *TODO*

    \section{Section 2}
    *TODO*

    \section{Section 3}
    *TODO*










%%%%%%%%%%%%%%%%%%%%%%%%%%%%%%%%%%%%%%%%%%%%%%%%%%%%%%%%%%%%%%%%%%%%%%%%%%%%%%%%%%%%%%%%%%%%%%%%%%%%%%%%%%%%%%%%%%%%%%%%
% project proposal

\cleardoublepage
\chapter{Project Proposal}

\input{proposal}










%%%%%%%%%%%%%%%%%%%%%%%%%%%%%%%%%%%%%%%%%%%%%%%%%%%%%%%%%%%%%%%%%%%%%%%%%%%%%%%%%%%%%%%%%%%%%%%%%%%%%%%%%%%%%%%%%%%%%%%%
% the glossary

\cleardoublepage
\chapter{Glossary}
\glsaddall
\printglossaries

\end{document}
